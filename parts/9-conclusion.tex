%% Макрос для заключения. Совместим со старым стилевиком.
\startconclusionpage

В данной работе было показано, что жесткая классификация вершин на
принадлежность активному модулю при разных пороговых значениях не согласована
и, соответственно, склонна к плохой интерпретации.  Поставлена задача
ранжирования вершин графа и задача оценки вероятности принадлежности вершин
активному модулю

Предложено три метода ранжирования вершин графа: оптимальный в среднем,
полу-эвристический и \emph{MCMC} ранжирования. Экспериментально было показано,
что все вышеперечисленные методы лучше, чем предложенные базовые методы. Так же
было показано, что \emph{MCMC} ранжирование работает лучше, чем все предложенные
методы.

Предложен метод на основе метода Монте-Карло по схеме марковских цепей.
Разработан критерий оценки ранжирования при условии, что известна вероятность
принадлежности каждой индивидуальной вершины активному модулю.


