%% Макрос для заключения. Совместим со старым стилевиком.
\startconclusionpage

В данной работе было показано, что жесткая классификация вершин по
принадлежности активному модулю при разных пороговых значениях не согласована
и, соответственно, склонна к плохой интерпретации.  Поставлена задача
ранжирования вершин графа и задача оценки вероятности принадлежности вершин
активному модулю.

Предложено три метода ранжирования вершин графа: оптимальный в среднем,
полуэвристический и ранжирование на основе вероятности вершины.
Экспериментально было показано, что все вышеперечисленные методы лучше, чем
предложенные базовые методы. Также было показано, что ранжирование на основе
вероятности работает лучше, чем все предложенные методы.

Предложен метод оценки вероятности вершин принадлежать в активный модуль на
основе метода Монте-Карло по схеме марковских цепей.  Разработан критерий
оценки ранжирования при условии, что известна вероятность принадлежности каждой
индивидуальной вершины активному модулю.
