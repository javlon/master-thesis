%% Макрос для введения. Совместим со старым стилевиком.
\startprefacepage

Сетевой анализ имеет много применений в биоинформатике. Он включает в себя анализ
сети коэкспрессии для кластеризации генов~\cite{Langfelder2008}, поиск
репортёрных метаболитов для метаболических процессов~\cite{Patil2005} и
стратификацию образцов опухолей на основе топологического расстояния между
соматическими мутациями в сетях взаимодействия генов~\cite{Hofree2013}. Главная
идея заключается в том, что, принимая во внимание связь между сущностями
(генами, метаболитами и т.д.), можно лучше интерпретировать соответствующие
исходные данные (экспрессию генов, концентрацию метаболитов и т.д.).

%Интегративные сетевые подходы обычно используются для интерпретации
%высокопроизводительных данных~\cite{Mitra2013}. Такие методы применяются во
%многих различных контекстах: в исследованиях ассоциации генома~\cite{Rossin2011},
%для выяснения механизмов метаболической регуляции~\cite{Jha2015}, для анализа
%оматических мутаций в раке~\cite{Leiserson2015} и т.д. Основная идея этих
%методов заключается в том, что рассмотрение внутренних связей (например, между
%белками, метаболитами или другими сущностями) может привести к более глубокому
%пониманию данных и соответствующих биологических процессов.

Информация о связности может использоваться несколькими способами. Простейший
анализ может включать ручное исследование соединений между входными
сигналами~\cite{Karnovsky2012}. Более сложные методы включают использование
соединений для анализа обогащения генов~\cite{Alexeyenko2012}, для
сравнения сетей~\cite{Ideker2012} и многое другое.

Один из типов сетевого анализа соответствует задачи восстановления
\emph{активного} или \emph{функционального модуля}. Цель этих методов -- найти
связную подсеть (модуль), которая обогащена индивидуально важными вершинами.
%Такой модуль, например, мог бы соответствовать сигнальному пути для сети
%белок-белковых взаимодействий~\cite{Dittrich2008a} или метаболического пути для
%метаболических сетей~\cite{Jha2015}.
%
%Один из наиболее хорошо разработанных и используемых подходов состоит в выборе
%связной подсети, которая лучше всего представляет модуль \emph{active} или
%\emph{functional}.
Первоначально эта концепция была предложена Айдекером и др.~\cite{Ideker2002}.
В ней была предложена метрика для подсчета подсети на основе данных экспрессии
генов с использованием эвристического метода для поиска сетей с наивысшим
рейтингом. С тех пор были разработаны несколько методов для решения этой
проблемы \emph{идентификации активного модуля}. Один из наиболее известных
методов, называемый \emph{BioNet}, был предложен Диттрихом
и др.~\cite{Dittrich2008a}.  Они предложили использовать схему подсчета подсети
с наивысшим правдоподобием, так что поиск наилучшей подсчетной подсети
соответствует решению задачи поиска подграфа максимального веса (\emph{Maximum
Weight Connected Subgraph}, \emph{MWCS}).  Хотя задача \emph{NP}-трудная, в той
же работе был предложен практически точный решатель.  Формулировка, основанная
на максимальном правдоподобии, в сочетании с точным решателем для
соответствующей задачи позволила добиться отличной производительности на
смоделированных данных и оказалась практически полезной на практике.

На данный момент все существующие
решатели~\cite{Ideker2002,Dittrich2008a,Alcaraz2012,Sergushichev2016} этой
задачи имеют недостаток немонотонной зависимости выходного модуля от порогового
значения.  Это означает, что при повторном запуске метода с более слабым
пороговым значением могут появляться не только некоторые новые вершины, но
некоторые также могут исчезнуть. Эта ситуация запутывает пользователя
и затрудняет интерпретацию результатов.

Есть еще один вопрос, который обычно не рассматривается в существующих методах
для задачи идентификации активного модуля, -- это уверенность в включении
отдельных вершин.  По конструкции в результирующих сетях вершины с высокой
индивидуальной значимостью связаны через менее значимые вершины. Возникает
вопрос, важнее ли вершины, включенные в модуль, вершин, не включенных в модуль, при том, что они имеют одинаковую индивидуальную значимость. Это особенно важно,
когда индивидуальная несущественно значимая вершина включена в модуль.
Неопределенность в этом аспекте может привести к неверному истолкованию данных
либо путем придания важности ложным вершинам, либо недостающим ключевым
вершинам.

Ранее Бейзер и др.~\cite{Beisser2012} предложили подход ресэмплинга складной
нож, где проблема с активным модулем решалась несколько раз для генерированных
входных данных.  Это позволяет вводить значения поддержки: сколько раз
определенная вершина или ребро были частью решения для повторно
ресэмплированных данных. Вычисленные значения поддержки могут затем
использоваться для выделения надежных сигналов от шума в результирующем модуле.

В общем случае вопрос о присвоении доверительных значений для отдельных вершин
можно рассматривать как проблему мягкой классификации. Вместо жесткой
классификации вершин в модуле или нет мы можем их ранжировать по значимости.

%В настоящей работе рассматривается формулировка задачи активного модуля
%в терминах монотонно-связного вершинного ранжирования.

В настоящей работе рассматривается формулировка задачи мягкой классификации
вершин.  Другими словами, это задача построения монотонно-связного вершинного
ранжирования. Это позволяет генерировать модули для нескольких пороговых
значений, которые согласуются друг с другом. Так же один из предложенных
методов оценивает вероятности принадлежности каждой вершины активному модулю.

%TODO: чтото про полуэвристический и оптимаольное в средном.

Метод, основанный на методе Монте-Карло по схеме марковских цепей (\emph{Monte
Carlo Markov Chain}, \emph{MCMC}), генерирует множество модулей из
апостериорного распределения.  С теоретической точки зрения такая выборка
позволяет найти вершинные вероятности с любой заданной точностью, выполняя
достаточные итерации цепи Маркова. Мы показываем, что этот метод также
практичен и может обеспечить высокую точность классификации в разумные сроки.

%Во-первых, в разделе~\ref{sec_formal_defs} мы формально определяем
%проблему и даем соответствующие определения. Затем в разделах~\ref{sec_optimal}
%и~\ref{sec_semiheuristic} мы предлагаем два метода решения проблемы: метод на
%основе грубой силы и полуэвристический метод, основанный на решении серии
%целочисленного линейного программирования (ILP) проблемы. Мы также определяем
%два базовых метода в разделе~\ref{sec_baseline}.  Наконец, в разделе
%~\ref{sec_experiments} мы сравниваем методы друг с другом и базовые методы
%в сгенерированных и реальных сетях.


